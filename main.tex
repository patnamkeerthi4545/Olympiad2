\documentclass{article}                           
\usepackage{gvv-book}                      
\usepackage{gvv}

\begin{document}
\begin{enumerate}

\item Let $f(x)=x^{n}+5x^{n-1}+3$, where $n>1$ is an integer. Prove that $f(x)$ cannot be expressed as the product of two nonconstant polynomials with integer coefficients.

\item Let $D$ be a point inside acute triangle $ABC$ such that $\angle ADB$ $=$ $\angle ACB$ $+$ $\pi/2$ and $AC \cdot BD = AD \cdot BC$. 
 
$(a)$ Calculate the ratio $(AB \cdot CD) / (AC \cdot B)$.
	
$(b)$ Prove that the tangents at $C$ to the circumcircles of $\triangle ACD$ and $\triangle BCD$ are perpendicular.

\item On an infinite chessboard, a game is played as follows. At the start, $n^{2}$ pieces are arranged on the chessboard in an $n$ by $n$ block of adjoining squares, one piece in each square. A move in the game is a jump in a horizontal or vertical direction over an adjacent occupied square to an unoccupied square immediately beyond. The piece which has been jumped over is removed.

Find those values of $n$ for which the game can end with only one piece remaining on the board.

\item For three points $P, Q, R$ in the plane, we define $m(PQR)$ as the minimum length of the three altitudes of $\triangle PQR$. (If the points are collinear, we set $m(PQR) = 0$.)


Prove that for points $A, B, C, X$ in the plane,

$m(ABC) \leq m(ABX) + m(AXC) + m(XBC)$.

\item Does there exist a function $f : N \rightarrow N$ such that $f(1) = 2$, $f(f(n)) = f(n) + n$ for all $n$ $\epsilon$ $N$, and $f(n) < f(n + 1)$ for all $n$ $\epsilon$ $N$ ?

\item There are $n$ lamps $L_0, \dots , L_{ n-1 }$ in a circle $(n > 1) $, where we denote $L_{n+k} = L_k$. (A lamp at all times is either on or off.) Perform steps $s_0, s_1, \dots $ as follows: at step $s_i$, if $L_{i-1}$ is lit, switch $L_i$ from on to off or vice versa, otherwise do nothing. Initially all lamps are on. Show that:

$(a)$ There is a positive integer $M(n)$ such that after $M(n)$ steps all the lamps are on again;

$(b)$ If $n = 2^{k}$, we can take $M(n) = n^{2}-1$; 


$(c)$ If $n = 2^{k}+1$, we can take $M(n) = n^{2}-n+1$.

\end{enumerate}
\end{document}  
